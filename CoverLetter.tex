%%%%%%%%%%%%%%%%%%%%%%%%%%%%%%%%%%%%%%%%%
% Long Lined Cover Letter
% LaTeX Template
% Version 1.0 (1/6/13)
%
% This template has been downloaded from:
% http://www.LaTeXTemplates.com
%
% Original author:
% Matthew J. Miller
% http://www.matthewjmiller.net/howtos/customized-cover-letter-scripts/
%
% License:
% CC BY-NC-SA 3.0 (http://creativecommons.org/licenses/by-nc-sa/3.0/)
%
%%%%%%%%%%%%%%%%%%%%%%%%%%%%%%%%%%%%%%%%%

%----------------------------------------------------------------------------------------
%	PACKAGES AND OTHER DOCUMENT CONFIGURATIONS
%----------------------------------------------------------------------------------------

\documentclass[10pt,stdletter,dateno,sigleft]{newlfm} % Extra options: 'sigleft' for a left-aligned signature, 'stdletternofrom' to remove the from address, 'letterpaper' for US letter paper - consult the newlfm class manual for more options

\usepackage{charter} % Use the Charter font for the document text

%\newsavebox{\Luiuc}\sbox{\Luiuc}{\parbox[b]{1.75in}{\vspace{0.5in}
%\includegraphics[width=1.2\linewidth]{logo.png}}} % Company/institution logo at the top left of the page
%\makeletterhead{Uiuc}{\Lheader{\usebox{\Luiuc}}}

\newlfmP{sigsize=5pt} % Slightly decrease the height of the signature field
%\newlfmP{addrfromphone} % Print a phone number under the sender's address
%\newlfmP{addrfromemail} % Print an email address under the sender's address
%\PhrPhone{Phone} % Customize the "Telephone" text
%\PhrEmail{Email} % Customize the "E-mail" text

%\lthUiuc % Print the company/institution logo

%----------------------------------------------------------------------------------------
%	YOUR NAME AND CONTACT INFORMATION
%----------------------------------------------------------------------------------------

\namefrom{Suryanarayan Mondal} % Name
%----------------------------------------------------------------------------------------
%	ADDRESSEE AND GREETING/CLOSING
%----------------------------------------------------------------------------------------
\greetto{Dear Prof. Vitaly Kudryavtsev,} % Greeting text
\nameto{Prof. Vitaly Kudryavtsev} % Addressee of the letter above the to address

\addrto{
 The University of Sheffield,\\
  UK.
}

%----------------------------------------------------------------------------------------

\begin{document}
\begin{newlfm}

%----------------------------------------------------------------------------------------
%	LETTER CONTENT
%----------------------------------------------------------------------------------------
  I am writing to apply for the position of Research Associate
  in The University of Sheffield with an emphasis in developing calibration procedure for DUNE Far detector using cosmic muons and Monte-Carlo Modelling and analysis of simulated data from DUNE far detector,
  that you advertised in INSPIRE-HEP Jobs. I did
  PhD in India Based Neutrino Observatory. My thesis title is
  ``Studies of Atmospheric Fluxes at the prototype iron calorimeter
  detector''. I am confident that
  my research interests make me an ideal candidate for this position.

  During my PhD, I got an opportunity to work on various studies
  related to RPC detectors and also cosmic ray physics. My journey in
  PhD commenced with commissioning and troubleshooting of the
  prototype ICAL made of the stack (12 layers) of large area RPCs
  ($~$\,2\,m\,$\times$\,2\,m). I was part of the team in assembling
  and troubleshooting the problems related to RPC detectors and their
  readout electronics. I worked to characterise the RPC detectors using the noise rate and muon triggered data. Later I used the muon data recorded with the
  detector stack to estimate the muon flux at the experimental site
  and its comparison with different models used in extensive air
  shower simulation. 
  I spent substantial time of my PhD for
  characterising the RPC detectors and their front-end modules, which
  helps in improving the design parameters of the detector as well as
  electronics.
  Along with the detector related hardware, I also worked
  on GEANT4
  simulation and incorporation of the detector parameters in
  simulation framework to reproduce the behaviour of the MC as close
  as DATA.
  My experience in data analysis will help in getting good
  physics results from the detectors. Especially, I will be useful in
  developing calibration techniques for the DUNE Far detector using cosmic muons or beam study as well doing Monte-Carlo Modelling and analysis of simulated data for DUNE Far detector.
   Apart from the thesis work I am working on improving the
  time/position resolution of the RPC detectors using
  Time-Over-Threshold. 
  
  My experience in detector commissioning, troubleshooting and data
  analysis would make me an excellent candidate for this position. I am
  looking forward to hearing from you soon.

%----------------------------------------------------------------------------------------
\closeline{Sincerely yours,} % Closing text
\end{newlfm}
\end{document}
