
\documentclass[12pt,a4]{article}

\usepackage[left=1.2in,right=0.8in,top=1in,bottom=1in]{geometry}
\usepackage{graphicx}
\usepackage{wrapfig}
\usepackage{float}
\usepackage{caption}
%\usepackage{subcaption}
%\usepackage{multicol}
%\usepackage{lmodern}
%\usepackage{textcomp}
\usepackage{hyperref}
\usepackage{tabularx}
\usepackage{array}
\usepackage{amsmath}
\usepackage{amssymb}
\usepackage{color}
\usepackage{mathtools}
\usepackage{slashed}
\usepackage{pdfpages}
\usepackage{listings}
\usepackage{siunitx}
\usepackage{latexsym}
\usepackage{multirow}
\usepackage{lineno}

%% \linenumbers




\begin{document}


\title{Statement of Research}
%% \date{}
\author{Suryanarayan Mondal}
\maketitle

\subsection*{Current Work}
My major interest lies in the design and development of particle detectors
and also commissioning of the large scale particle physics experiment to
study the fundamental nature of the elementary particles. I had a great
opportunity to work in the commissioning of the two different prototypes
of ICAL-experiment in IICHEP-Madurai. The first prototype stack was built
using 12 layers of Resistive Plate Chamber detectors of 2 m � 2 m area.
%% The second one was built using eleven layers
%% of iron plates with a dimension of 4 m � 4 m � 0.056 m stacked with an interlayer distance of 4.5 cm.
%% RPCs are inserted in between iron layers (name of the prototype is called mini-ICAL). These stacks
%% were used to study the long-term performance of the RPCs, the front-end and back-end electronics
%% made in Indian industry. I was associated with the commissioning, troubleshooting the detector
%% stack and physics analysis using the muon data recorded using prototype stacks.
%% After the successful commissioning of the prototype stacks, I was mainly involved in the data
%% analysis of noise rate and muon data recorded by prototype stack. I worked on characterising the
%% RPCs and electronics using the muon data and noise rate data. I got expertise in identifying the
%% source of problems in RPCs and electronics by periodic analysis of the muon data. The experience
%% I gained in troubleshooting would be vital knowledge for any small scale and large scale particle
%% physics experiments. The characterisation of the hardware elements helped to qualify indigenous
%% made RPCs and electronics.
%% Besides the hardware related work, my PhD title is to do physics analysis using muon data. The
%% recorded muon data from the prototype stack was used to measure the integral intensity of vertical
%% muons, zenith angle dependent muon flux and azimuthal dependent muon flux at different zenith
%% angle bins. The observed muon flux at the experimental site was compared with phenomenological
%% models using CORSIKA and HONDA predictions. The suitable interaction model to reproduce
%% the experimental data was summarised. The above-mentioned physics outputs are published in
%% peer-reviewed journals.
%% In addition to PhD work, I am working to improve the position and time resolution of the RPC
%% detectors used in the mini-ICAL prototype. Time Over Threshold information of RPC signals are
%% incorporated in the offline analysis to have better position and time resolution. I also worked in an
%% algorithm to reconstruct muons in the magnetic field.

\subsection*{Interest}
%% As I gained a lot of experience in the detector R&D, simulation and data analysis during my PhD.
%% I would like to continue my involvement in R&D and characterisation of different type of particle
%% detectors using cosmic muons. Especially, I would like to work on optimising the design parameters
%% to enhance the sensitivity of the liquid argon detectors. Other than detector R&D, I would like to
%% involve in physics data analysis, where I desire to work on particle reconstruction techniques.

\end{document}


