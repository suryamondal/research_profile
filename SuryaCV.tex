\documentclass[12pt]{article}
\usepackage{graphicx}
\usepackage{times}
\usepackage{wasysym}
\usepackage[a4paper,left=0.8in,right=1in,top=0.5in,bottom=0.85in]{geometry}
\usepackage{tfrupee}
\usepackage{xcolor}
%% \usepackage[colorlinks]{hyperref}
%% \usepackage{url}
%% \usepackage{enumitem}
\usepackage{enumerate}
\usepackage{hyperref}
\let\iint\relax
\let\iiint\relax
\let\iiiint\relax
\let\idotsint\relax
\usepackage{lipsum}
\usepackage{fancyhdr}
\usepackage{siunitx}
\usepackage{latexsym}
\usepackage{txfonts}
\setlength{\parindent}{4em}
\setlength{\parskip}{0.5em}
\usepackage[none]{hyphenat}
\usepackage{ragged2e}
%% Define variables here

\begin{document}
\definecolor{gray(x11gray)}{rgb}{0.75, 0.75, 0.75}
\definecolor{lavender(web)}{rgb}{0.9, 0.9, 0.98}
\hypersetup{urlcolor=DarkBlue}
\pagestyle{plain}
\flushleft
{\bf {CURRICULUM VITAE }}\\
%\noindent\makebox[\linewidth]{\rule{\paperwidth}{0.2pt}}
\noindent\rule{18cm}{0.2pt}
\vspace{0.05cm}
\noindent

\begin{minipage}{0.8\textwidth}
 {\bf{Suryanarayan Mondal}} \\[0.4em]
 Postdoctoral Researcher,\\
 Department of Physics,
 University of Pisa,\\
 Largo Bruno Pontecorvo 3,\\
 Pisa, Italy 56127 \\
 E-mail : suryamondal@gmail.com\\
 Mobile : +91-9732993380 \\[0.1em]
\end{minipage}% This must go next to `\end{minipage}`
%% \begin{minipage}{0.5\textwidth}
%%   \includegraphics[width=0.45\linewidth]{./SMA454.jpg}
%% \end{minipage}

\vspace{0.6cm}
{\bf {Permanent Address }}\\
\begin{minipage}{0.8\textwidth}
  \vspace{0.3cm}
  Chandpur Uttarpara,\\
  Jalaberia (PO),\\
  South 24 Parganas,\\
  West Bengal--743338,\\
  India
\end{minipage}

\vspace{2cm}
\colorbox{gray!40}{\begin{minipage}{17.5cm}
\bf { EDUCATION} 
\end{minipage} }

\vspace{0.4cm}
\begin{tabular}{p{3cm} p{14cm} }

  {\emph{PhD} (2021)} &  \emph{Thesis:} `Multiplicity of muon in $2$\,m\,$\times$\,2\,m detector and charge ratio of cosmic muon at Madurai'\\
  \vspace{0.2cm}
  & Homi Bhabha National Institute, Anushaktinagar, Mumbai, India. \\
  & \\
  {\emph{MSc} (2014)} & Department of Physics, Indian Institute of Technology Madras, Chennai, Tamil Nadu, India.\\
  & \\
  {\emph{BSc} (2012)} & Department of Physics, Ramakrishna Mission Vidyamandira (Affiliated to Calcutta University), Howrah, West Bengal, India.\\
  & \\
  {\emph{HSC} (2009)} & Majilpur J. M. Training School (Affiliated to West Bengal Council of Higher Secondary Education), South 24 Parganas, West Bengal, India. \\
  & \\
  {\emph{SSC} (2007)} & Majilpur J. M. Training School (Affiliated to West Bengal Board of Secondary Education), South 24 Parganas, West Bengal, India. \\
  & \\
  {\emph{Primary} (2001)} & Chandpur Free Pay School (Affiliated to West Bengal Board of Primary Education), South 24 Parganas, West Bengal, India.
  
\end{tabular}

\pagebreak
\vspace{0.5cm}
\colorbox{gray!40}{\begin{minipage}{17.5cm}
\bf {RESEARCH EXPERIENCE } 
\end{minipage} }

\begin{minipage}{1.05\textwidth}
\vspace{0.4cm}
\begin{tabular}{p{2.5cm} p{14.5cm} }
  {\emph{Postdoctoral Researcher}} & University of Pisa, Largo Bruno Pontecorvo, Pisa, Italy 56127 (Oct 2021 -- Oct 2024)

  \begin{itemize}
  \item I am now involved in the Belle II Collaboration, more
    specifically, in upgrade group and the Silicon Vertex Detector (SVD).
  \item An all-pixel vertex detector (VTX) is under proposal as a replacement of
    the present one. With smaller pixel size, it is capable of higher
    hit-occupancy in sensors and better vertex resolution.
    I am studying the physics performance using simulation in a
    few benchmark decay channels, $B^{0} \rightarrow D^{*-}\mu^{+}\nu_{\mu}$,
    $D^{*-} \rightarrow \overline{D}^{0} \pi^{-}_{\text{soft}}$,
    $\overline{D}^{0} \rightarrow K^{+}[\pi^{-}\pi^{+}]\pi^{-}$.
    Along with the unknown flight length of $B^0$ and combinatorials due to
    negligible $D^{*}$ flight length, low momentum of $\pi_{\text{soft}}$ makes
    the reconstruction of these channels rely heavily on the vertex
    detector. Along with excellent vertexing capabilities, VTX performs very well
    in terms of efficiency. 
    This performance study is important for the Belle II Upgrade Program and will be
    published in the next Conceptual Design Report (CDR).

    The upgrade work is a continuous progress. The detector geometry is
    periodically in testing with updated material budgets and using improved
    background predictions. 

  \item I developed on a new algorithm to minimise the effect of background in the
    track reconstruction of the current SVD in Belle II.
    It exploits the time of SVD clusters event-by-event, grouping the clusters
    depending on the contribution by different collisions.
    Only the best group of
    clusters is chosen for the reconstruction. Using this method,
    the rate of fake-tracks is reduced significantly while keeping the efficiency
    unchanged.

    This module is now part of the official release of the Belle II software and under
    test in data.
    This method will play an important role at the target luminosity helping to set the limit
    of SVD hit-occupancy to \SI{\sim6}{\%}, which is higher than the one expected from the background
    extrapolation \SI{\sim4.7}{\%}, increasing the safety margin. Additionally, it is now
    used as helper to the SVD cluster-time calibration which was previously significantly effected
    by the background hits biasing the calibration.

  \item I improved the quality of the cluster time calibration on data by implementing a
    a new calibration that removes the cluster-size dependant bias on cluster-time.

  \item Alongside, I contributed to the Belle II software in order to improve its quality.
    I managed to find and resolve the issue of high execution time consumed by
    the SVD cluster-time calibration process. The issue was quite technical.
    The merging of histograms from large number of ROOT files consumes time and
    SVD requires more than thousand histograms during calibration.
    I managed to reduce the number of histograms by using multi-dimensional
    histograms which in turns reduces
    the execution time dramatically from days to minutes.
  \end{itemize}

\end{tabular}
\end{minipage}

\pagebreak
\vspace{0.5cm}
\colorbox{gray!40}{\begin{minipage}{17.5cm}
\bf {RESEARCH EXPERIENCE } 
\end{minipage} }

\begin{minipage}{1.05\textwidth}
\vspace{0.4cm}
\begin{tabular}{p{2.5cm} p{14.5cm} }
  {\emph{Research Scholar}} & Tata Institute of Fundamental Research, Mumbai, India (Aug 2014 -- Aug 2021)
  \begin{itemize}
  \item I was involved in India-based Neutrino
    Observatory (INO). My works were solely contributed towards
    the proposed Iron Calorimeter (ICAL) detector. This future
    underground facility is going to be dedicated for the study of the
    oscillation parameters from atmospheric neutrinos along with
    the mass ordering. Resistive Plate Chambers (RPCs) with glass
    electrodes are chosen as the sensitive detector in INO-ICAL to
    sense the signature of the muons produced in the charged-current
    interaction of neutrinos. Many prototype detector stacks were thus
    planned to study the performance of the RPCs and the electronics
    along with the data acquisition systems. I clocked a significant presence
    in commissioning two prototypes, gaining experience and contributing
    in some key aspects.
  \item Involved in the mechanical facets of commissioning the detectors.
  \item Formulated and implemented a easy-to-use system to estimate the
    leak tightness of RPC gaps with high precision, which is in use at
    collaborating institutes and industries.
  \item Developed an algorithm to fetch events with multiple tracks
    in the detector. Also isolated the events occurring due to the
    random coincidences in order to test the CORSIKA simulation.
  \item As the part of the mini-ICAL team, I was stationed at the complex
    of \emph{Saint-Gobain Glass India Pvt.Ltd.} for 2\,months
    supervising the production, especially stream-lining the quality-control
    of first batch of RPCs
    to be used in the prototype of ICAL.
  \item Worked in developing a method to reconstruct muon momentum
    from the data obtained in the magnetised mini-ICAL detector
    consisting of 10 layers of RPCs for the measurement of the charge
    ratio of low energy muon at the earth surface.
  \item Grasp in GEANT4, CERN-ROOT, CORSIKA.
  \end{itemize}

\end{tabular}
\end{minipage}

\pagebreak
\vspace{0.4cm}
\colorbox{gray!40}{\begin{minipage}{16.2cm}
\bf { STATEMENT OF RESEARCH INTERESTS} 
\end{minipage} }
%% \vspace*{0.0cm}
%% \hspace{0.5cm}

\begin{justify}

  Currently I am at the University of Pisa since October 2021 and should end in October 2024.
  This position will be a great opportunity for me to plan and continue my research ahead.
  My main research interest rests in the high-energy physics, especially
  in the design, development and commissioning of the detector components
  along with the analysis of acquired data.\\
  \hspace*{0.6cm}
  Modern particle detector setups are complex. These
  setups consist of various kinds of particle detectors.
  The GEANT4 simulation toolkit is proven to be effective in these
  cases, simulating the properties and response of the various
  components. This simulation provides a crucial role in understanding
  the feasibility and the physics potential of the experiment.
  Once these detector setups are commissioned, it is hard to access the
  individual components. Hence much interest is given towards the
  characterisation of each detector before integrating in the setup.
  This process is generally repetitive and thus should require least
  presence of human and minimal time.
  %%  I am interested in designing
  %% user-friendly test rigs to map the necessary traits of a detector.\\
  %% \hspace*{0.6cm}
  As the detectors grow larger, the number of data-channels also
  increases accordingly. So processing the events in order to extract
  the useful information by avoiding noises
  becomes more challenging. Along with this, an algorithm to reconstruct
  track parameters is required to optimise specific to the detector
  setup.\\
  \hspace*{0.6cm}
  My past involvement with the {\it India-based Neutrino Observatory (INO)} project
  let me work with a few prototype setups; especially
  a 12-layer RPC stack and a magnetised
  mini-ICAL, both are of tracker type. Both the setups gave me plenty of
  opportunity to gain knowledge on the aforesaid topics. Apart from
  that I learned about the difficulties and hold-ups while commissioning
  a detector setup and also gained experience of the solutions.
  I had developed the leak test system from scratch and it is now
  used by the whole INO collaboration. Similarly during the
  commissioning of mini-ICAL, I had solved many challenging problems; one of
  them even saved us from dismantling the whole magnet.
  In summary, I gained experience in both software and hardware during this time.\\
  %% ,
  %% e.g., how to unbolt the aluminium strips from the commissioned
  %% miniICAL magnet system without dismantling the whole magnet.
  %% These aluminium strips were bolted to the
  %% magnet system for the smooth movement of the RPC trays in each layer,
  %% but due to the magnetic field, the whole system shrinks beyond our
  %% estimation, buckling those strips
  %% .\\
  %% During commissioning of RPC detector at miniICAL, it was found that
  %% few RPC can not hold pressure more than 10mm of water column, which is
  %% necessary for stable operation of the RPC. I had temporarily
  %% solved that button popup issue by installing suitable mylar balloon
  %% over the RPC trays.\\
  %% \hspace*{0.6cm}
  %% My concept of a FTIR detector setup to
  %% monitor the INO gas system would have been much cheaper than a
  %% commercial one, but could not finish due to shortage of the components.\\
  \hspace*{0.6cm}
  My current involvement in the Belle II experiment gave me additional
  opportunity to boost my knowledge in the accelerator based experiments.
  Here I got to learn about the schemes and frameworks of the software tools
  used along with the physics goals of the experiment.
  An all-pixel vertex detector is under proposal as a replacement of
  the present one in the Belle II detector. With smaller pixel size, it is
  capable of higher occupancy and vertex resolution. I am highly involved
  in this group and contributing in assessing the physics performance of the
  new setup using simulation in a few benchmark decay channels.
  I am also working in the existing Silicon Vertex Detector of Belle II to
  reduce the effect of background in track finding. I devised a technique
  of groping the position clusters based on time to suppress the clusters
  contributed by background collision. This reduces the rate of fake-tracks
  significantly.\\
  \hspace*{0.6cm}
  I carried forward my experience and skills from INO to Belle II, enriching it
  significantly and want to spend more time in both SVD and the vertex detector (VTX) upgrade.
  \\
  $\blacksquare$ Large uncertainity in the prediction of background at the peak luminosity is
  going to effect the tracking performance and the the latest estimate reduced the
  tracking effiency to \SI{91}{\%}. The effort to recover the efficiency is fundamental for the
  physics program.
  I want to explore more in the tracking of the SVD believing in improving
  the track finding efficiency keeping the rate of fake tracks low. Now the method of
  grouping is not used at its full potential and there are large room for improvement;
  one of which is to use MVA to better filter groups selected for tracking.
  \\
  $\blacksquare$ I will be involved greatly in the in the upgrade of the vertex detector,
  not only in the benchmarking and software developements, also in the hardware activities.
  I can contribute to various aspects of the design and testing of the sensors for next few
  years targeting the Belle II Long Shutdown 2 in 2027.
  \\
  \hspace*{0.6cm}
  The Belle II team in Pisa is excellent in both work and motivation.
  It will be great to continue work here.

\end{justify}

\pagebreak
\vspace{0.4cm}
\colorbox{gray!40}{\begin{minipage}{16.2cm}
    \bf {LIST OF PUBLICATIONS} 
\end{minipage} }
\begin{justify}
%% \vspace{0.4cm}
\begin{enumerate}[a.]
\item Publications:
  \begin{enumerate}[1.]
    %% \item I~Adachi~et~al. \emph{}, \href{}{}
    %% \item I~Adachi~et~al. \emph{}, \href{}{}
    %% \item I~Adachi~et~al. \emph{}, \href{}{}
    %% \item I~Adachi~et~al. \emph{}, \href{}{}
    %% \item I~Adachi~et~al. \emph{}, \href{}{}
    %% \item I~Adachi~et~al. \emph{}, \href{}{}
    %% \item I~Adachi~et~al. \emph{}, \href{}{}
    %% \item I~Adachi~et~al. \emph{}, \href{}{}
    %% \item I~Adachi~et~al. \emph{A test of lepton flavour universality with a measurement of $R(D^{*})$ using hadronic tagging at the Belle II experiment}, \href{}{}
    %% \item I~Adachi~et~al. \emph{Search for $e^+ e^{\mathstrut -} \to \eta_b(1S)\omega$ and $e^+ e^{\mathstrut -} \to \chi_{b0}(1P)\omega$ at $\sqrt{s} = 10.745$ GeV with Belle II}, \href{}{}
  \item I~Adachi~et~al. \emph{Precise measurement of the $D_s^+$ lifetime at Belle II}, \href{https://doi.org/10.1103/PhysRevLett.131.171803}{Physical Review Letters 131 (2023) 171803}
    %% \item I~Adachi~et~al. \emph{Measurement of the energy dependence of the $e^+ e^{\mathstrut -}\to B\bar{B}$, ${B}\bar{B}{}^*$ and ${B}{}^*\bar{B}{}^*$ cross sections at Belle II}, \href{}{}
    %% \item I~Adachi~et~al. \emph{Tests of light-lepton universality in angular asymmetries of hadronically tagged $B^0 \to D^{*-} \ell^+ \nu$ decays at Belle II}, \href{}{}
    %% \item I~Adachi~et~al. \emph{Measurement of branching-fraction ratios and $CP$ asymmetries and constraint of CKM angle $\phi_3$ with $B^{\pm} \to D_{CP\pm} K^{\pm}$ decays at Belle and Belle II}, \href{}{}
    %% \item I~Adachi~et~al. \emph{Measurement of asymmetries and branching-fraction ratios for $B^{\pm} \to D K^{\pm}$ and $B^{\pm} \to D\pi^{\pm}$ with $D \to K_S^0 K^{\pm} \pi^{\mp}$ using Belle and Belle II data}, \href{}{}
    %% \item I~Adachi~et~al. \emph{Search for a long-lived scalar particle in $b \to s$ transitions at the Belle II experiment}, \href{}{}
  \item I~Adachi~et~al. \emph{Measurement of the $\tau$ lepton mass with the Belle II experiment}, \href{https://doi.org/10.1103/PhysRevD.108.032006}{Physical Review D 108 (2023) 032006}
    %% \item I~Adachi~et~al. \emph{Measurement of branching fractions and direct $CP$ asymmetries of $B \to K \pi$ and $B \to \pi \pi$ decays in 2019-2022 Belle II data}, \href{}{}
    %% \item I~Adachi~et~al. \emph{Measurement of time-dependent $CP$ violation in $B^0 \to K_S^0 K_S^0 K_S^0$ using 2019-2022 Belle II data}, \href{}{}
    %% \item I~Adachi~et~al. \emph{Determination of $|V_{cb}|$ using $\bar{B}{}^0 \to D^{*+} \ell^{\mathstrut -} \bar{\nu}{}_{\ell}$ decays with Belle II}, \href{}{}
  \item I~Adachi~et~al. \emph{Measurement of $CP$ violation in $B^0 \rightarrow K^0_{S} \pi^0$ decays at Belle II}, \href{https://doi.org/10.1103/PhysRevLett.131.111803}{Physical Review Letters 131 (2023) 111803}
  \item I~Adachi~et~al. \emph{Measurement of $CP$ asymmetries in $B^0 \rightarrow \phi K^0_{S}$ decays with Belle~II}, \href{https://doi.org/10.1103/PhysRevD.108.072012}{Physical Review D 108 (2023) 072012}
  \item I~Adachi~et~al. \emph{Novel method for the identification of the production flavor of neutral charmed mesons}, \href{https://doi.org/10.1103/PhysRevD.107.112010}{Physical Review D 107 (2023) 112010}
  \item I~Adachi~et~al. \emph{Search for a $\tau^+\tau^-$ resonance in $e^{+}e^{-}\rightarrow \mu^{+}\mu^{-} \tau^+\tau^-$ events with the Belle~II experiment}, \href{https://doi.org/10.1103/PhysRevLett.131.121802}{Physical Review Letters 131 (2023) 121802}
  \item G.~Batignani~et~al., \emph{Simulation of an all-layer monolithic pixel vertex detector for the Belle II upgrade in 16th Pisa Meeting on Advanced Detectors}, \href{https://doi.org/10.1016/j.nima.2022.167616}{NIMA 1045 (January 2023) 167616}
  \item Suryanarayan~Mondal~et~al. \emph{Study of Particle Multiplicity of Cosmic Ray Events using 2\,m\,$\times$\,2\,m Resistive Plate Chamber Stack at IICHEP-Madurai}, \href{https://doi.org/10.1007/s10686-020-09685-6}{Experimental Astronomy 51 (2021) 17--32}
  \item Suryanarayan~Mondal~et~al. \emph{Leak test of Resistive Plate Chamber gap by monitoring absolute pressure}, \href{https://doi.org/10.1088/1748-0221/14/04/P04009}{Journal of Instrumentation 14 (April 2019) P04009}
  \item S.~Mondal~et~al., \emph{Leak Rate Estimation of a Resistive Plate Chamber Gap by Monitoring Absolute Pressure in 13th Workshop on Resistive Plate Chambers and Related Detectors (RPC2016)}, \href{https://doi.org/10.1088/1748-0221/11/11/C11009}{Journal of Instrumentation 11 (Nov 2016) C11009}
  \item  G.~Majumder~et~al., \emph{Development of a Resistive Plate Chamber with heat strengthened glass in 13th Workshop on Resistive Plate Chambers and Related Detectors (RPC2016)}, \href{https://doi.org/10.1088/1748-0221/11/09/C09019}{Journal of Instrumentation 11 (2016) C09019}
  \end{enumerate} 
\end{enumerate} 
\begin{enumerate}[b.]
\item Proceedings
  \begin{enumerate}[1.]
  \item Suryanarayan~Mondal~et~al., \emph{Cosmic Muon Momentum Spectra at Madurai in XXIV DAE High Energy Physics Symposium}, \href{https://doi.org/10.1007/978-981-19-2354-8_134}{Springer Proceedings in Physics 277 (October 2022) 743--747}
  \item Suryanarayan~Mondal~et~al., \emph{Study of Particle Multiplicity by 2\,m\,$\times$\,2\,m Resistive Plate Chamber Stack at IICHEP-Madurai in XXIII DAE High Energy Physics Symposium}, \href{https://doi.org/10.1007/978-981-33-4408-2_172}{Springer Proceedings in Physics 261 (May 2021) 1155--1158}
  \item G.~Majumder and S.~Mondal, \emph{Design, construction and performance of magnetised mini-ICAL detector module in The 39th International Conference on High Energy Physics (ICHEP2018)}, \href{https://doi.org/10.22323/1.340.0360}{Proceedings of Science 340 (2019) 360}
  \item Suryanarayan~Mondal~et~al., \emph{Estimation of Leak of a Resistive Plate Chamber by Monitoring Absolute Pressure in XXII DAE High Energy Physics Symposium}, \href{https://doi.org/10.1007/978-3-319-73171-1_207}{Springer Proceedings in Physics 203 (May 2018) 851-853}
  \item G.~Majumder~et~al., \emph{Development of a Resistive Plate Chamber with Heat Strengthened Glass in XXII DAE High Energy Physics Symposium}, \href{https://doi.org/10.1007/978-3-319-73171-1_135}{Springer Proceedings in Physics 203 (2018) 575-578}
  \item S.~Pethuraj~et~al., \emph{Measurement of Angular Distribution and Integrated Flux of Cosmic Ray Muons Using 2\,m$\times$2\,m RPC Stack at IICHEP Madurai in XXII DAE High Energy Physics Symposium}, \href{https://doi.org/10.1007/978-3-319-73171-1_205}{Springer Proceedings in Physics 203 (2018) 845-846}
  \end{enumerate}
\end{enumerate}
\end{justify}

\newpage

\colorbox{gray!40}{\begin{minipage}{16.2cm}
\bf {Conferences/Workshops} 
\end{minipage} }
\begin{minipage}{1.0\textwidth}
 \vspace{0.4cm}
\begin{enumerate}
  
\item Attended \emph{16$^{\text{th}}$ Topical Seminar on Innovative Particle and Radiation Detectors (IPRD23) held at Siena, Italy during 25--29 September 2023.} \\
  {\bf{Talk:}} \textsc{The Silicon Vertex Detector of the Belle II Experiment}.

\item Attended \emph{XXIV DAE-BRNS High Energy Physics Symposium}  held at NISER, Bhubaneswar, India during 14-18 December, 2020. \\
  {\bf{Poster:}} \textsc{Correlation of muons arrival times from two different cosmic showers}.\\
  {\bf{Talk:}} \textsc{Cosmic muon momentum spectra at Madurai}.

\item Attended \emph{XXIII DAE-BRNS High Energy Physics Symposium}  held at IIT Madras, Chennai, India during 10-14 December, 2018. \\
  {\bf{Poster:}} \textsc{Muon Multiplicity in $2$\,m\,$\times$\,2\,m RPC and comparison with CORSIKA simulation}.

\item Attended \emph{National Symposium on Particles, Detectors \& Instrumentation (NSPDI 2017)} held at Tata Institute of Fundamental Research, Mumbai,  India during 4-7 October, 2017. \\
  {\bf{Poster:}} \textsc{Estimation of Leak of a Resistive Plate Chamber by Monitoring Absolute Pressure}.

\item Attended \emph{13th Workshop on Resistive Plate Chambers and
Related Detectors (RPC2016)} held at Ghent University, Ghent,
  Belgium during 22--26 February, 2016.\\
  {\bf{Poster:}} \textsc{Leak Rate Estimation of a Resistive Plate Chamber Gap by Monitoring Absolute Pressure}.
  
\item Attended \emph{XXII DAE-BRNS High Energy Physics Symposium}  held at University of Delhi, Delhi, India during 12-16 December, 2016. \\
  {\bf{Poster:}} \textsc{Estimation of Leak of a Resistive Plate Chamber by Monitoring Absolute Pressure}.
  
\item Attended the course of \emph{Japan-Asia Youth Exchange program in Science (SAKURA Exchange Program in Science)} administered by Japan Science and Technology Agency and held at Osaka university, Osaka, Japan during 29--04 December, 2015. 



  
  

\end{enumerate}
\end{minipage}

\newpage
\vspace{0.6cm}
\colorbox{gray!40}{\begin{minipage}{17.5cm}
\bf {Personal Details}
\end{minipage} }

\begin{minipage}{1.05\textwidth}
\vspace{0.4cm}
\begin{tabular}{ r l}
  {\emph {Mother:}} & Minakshi Jana (Mondal) \\[0.2cm]
  {\emph {Father:}} & Lakshmi Narayan Mondal \\[0.2cm]
  {\emph {Date of Birth:}} & 12 March 1990 \\[0.2cm]
  {\emph {Place of Birth:}} & Jaynagar Majilpur, West Bengal, India \\[0.2cm]
  {\emph {Hobby:}} & Trekking \\[0.2cm]
\end{tabular}
\end{minipage} 

\end{document}
